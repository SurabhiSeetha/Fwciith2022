\documentclass[journal,12pt,twocolumn]{IEEEtran}

\usepackage[utf8]{inputenc}
\usepackage{kvmap}
\usepackage{graphics} 

\usepackage{setspace}
\usepackage{gensymb}

\singlespacing


\usepackage{amsthm}

\usepackage{mathrsfs}
\usepackage{txfonts}
\usepackage{stfloats}
\usepackage{bm}
\usepackage{cite}
\usepackage{cases}
\usepackage{subfig}

\usepackage{longtable}
\usepackage{multirow}

\usepackage{enumitem}
\usepackage{mathtools}
\usepackage{steinmetz}
\usepackage{tikz}
\usepackage{circuitikz}
\usepackage{verbatim}
\usepackage{tfrupee}
\usepackage[breaklinks=true]{hyperref}
\usepackage{graphicx}
\usepackage{tkz-euclide}
\usepackage{float}

\usetikzlibrary{calc,math}
\usepackage{listings}
    \usepackage{color}                                            %%
    \usepackage{array}                                            %%
    \usepackage{longtable}                                        %%
    \usepackage{calc}                                             %%
    \usepackage{multirow}                                         %%
    \usepackage{hhline}                                           %%
    \usepackage{ifthen}                                           %%
    \usepackage{lscape}     
\usepackage{multicol}
\usepackage{chngcntr}

\DeclareMathOperator*{\Res}{Res}

\renewcommand\thesection{\arabic{section}}
\renewcommand\thesubsection{\thesection.\arabic{subsection}}
\renewcommand\thesubsubsection{\thesubsection.\arabic{subsubsection}}

\renewcommand\thesectiondis{\arabic{section}}
\renewcommand\thesubsectiondis{\thesectiondis.\arabic{subsection}}
\renewcommand\thesubsubsectiondis{\thesubsectiondis.\arabic{subsubsection}}


\hyphenation{op-tical net-works semi-conduc-tor}
\def\inputGnumericTable{}                                 %%

\lstset{
%language=C,
frame=single, 
breaklines=true,
columns=fullflexible
}
\begin{document}


\newtheorem{theorem}{Theorem}[section]
\newtheorem{problem}{Problem}
\newtheorem{proposition}{Proposition}[section]
\newtheorem{lemma}{Lemma}[section]
\newtheorem{corollary}[theorem]{Corollary}
\newtheorem{example}{Example}[section]
\newtheorem{definition}[problem]{Definition}

\newcommand{\BEQA}{\begin{eqnarray}}
\newcommand{\EEQA}{\end{eqnarray}}
\newcommand{\define}{\stackrel{\triangle}{=}}
\newcommand\hlight[1]{\tikz[overlay, remember picture,baseline=-\the\dimexpr\fontdimen22\textfont2\relax]\node[rectangle,fill=blue!50,rounded corners,fill opacity = 0.2,draw,thick,text opacity =1] {$#1$};}
\bibliographystyle{IEEEtran}
\providecommand{\mbf}{\mathbf}
\providecommand{\pr}[1]{\ensuremath{\Pr\left(#1\right)}}
\providecommand{\qfunc}[1]{\ensuremath{Q\left(#1\right)}}
\providecommand{\sbrak}[1]{\ensuremath{{}\left[#1\right]}}
\providecommand{\lsbrak}[1]{\ensuremath{{}\left[#1\right.}}
\providecommand{\rsbrak}[1]{\ensuremath{{}\left.#1\right]}}
\providecommand{\brak}[1]{\ensuremath{\left(#1\right)}}
\providecommand{\lbrak}[1]{\ensuremath{\left(#1\right.}}
\providecommand{\rbrak}[1]{\ensuremath{\left.#1\right)}}
\providecommand{\cbrak}[1]{\ensuremath{\left\{#1\right\}}}
\providecommand{\lcbrak}[1]{\ensuremath{\left\{#1\right.}}
\providecommand{\rcbrak}[1]{\ensuremath{\left.#1\right\}}}
\theoremstyle{remark}
\newtheorem{rem}{Remark}
\newcommand{\sgn}{\mathop{\mathrm{sgn}}}
\providecommand{\abs}[1]{\left\vert#1\right\vert}
\providecommand{\res}[1]{\Res\displaylimits_{#1}} 
\providecommand{\norm}[1]{$\left\lVert#1\right\rVert$}
%\providecommand{\norm}[1]{\lVert#1\rVert}
\providecommand{\mtx}[1]{\mathbf{#1}}
\providecommand{\mean}[1]{E\left[ #1 \right]}
\providecommand{\fourier}{\overset{\mathcal{F}}{ \rightleftharpoons}}
%\providecommand{\hilbert}{\overset{\mathcal{H}}{ \rightleftharpoons}}
\providecommand{\system}{\overset{\mathcal{H}}{ \longleftrightarrow}}
	%\newcommand{\solution}[2]{\textbf{Solution:}{#1}}
\newcommand{\solution}{\noindent \textbf{Solution: }}
\newcommand{\cosec}{\,\text{cosec}\,}
\providecommand{\dec}[2]{\ensuremath{\overset{#1}{\underset{#2}{\gtrless}}}}
\newcommand{\myvec}[1]{\ensuremath{\begin{pmatrix}#1\end{pmatrix}}}
\newcommand{\mydet}[1]{\ensuremath{\begin{vmatrix}#1\end{vmatrix}}}
\numberwithin{equation}{subsection}
\makeatletter
\@addtoreset{figure}{problem}
\makeatother
\let\StandardTheFigure\thefigure
\let\vec\mathbf
\renewcommand{\thefigure}{\theproblem}
\def\putbox#1#2#3{\makebox[0in][l]{\makebox[#1][l]{}\raisebox{\baselineskip}[0in][0in]{\raisebox{#2}[0in][0in]{#3}}}}
     \def\rightbox#1{\makebox[0in][r]{#1}}
     \def\centbox#1{\makebox[0in]{#1}}
     \def\topbox#1{\raisebox{-\baselineskip}[0in][0in]{#1}}
     \def\midbox#1{\raisebox{-0.5\baselineskip}[0in][0in]{#1}}
\vspace{3cm}
\title{\textbf{Optimization-Advanced} }
\author{Surabhi Seetha}
\maketitle
\newpage
\bigskip
\renewcommand{\thefigure}{\theenumi}
\renewcommand{\thetable}{\theenumi}
Get Python code for the figure from 
\begin{lstlisting}
https://github.com/SurabhiSeetha/Fwciith2022/tree/main/Assignment%201/codes/src
\end{lstlisting}
Get LaTex code from
\begin{lstlisting}
https://github.com/SurabhiSeetha/Fwciith2022/tree/main/avr%20gcc
\end{lstlisting}
%
\section{Question}
\centering
\textbf{Q(24), Class - 12, CBSE Paper, 2013}\\
\vspace{0.5cm}
\raggedright{Show that the height of the cylinder of maximum volume that can be inscribed in a sphere of radius R is $\frac{2R}{\sqrt{3}} $. Also find the maximum volume.}\\
\centering
\section{Solution}
\vspace{0.25cm} 
\raggedright
Let \textbf{h} be the height of the Cylinder\\
\vspace{0.2cm}
Let \textbf{r} be the radius of the Cylinder\\
\vspace{0.2cm}
Let \textbf{R} be the radius of the Sphere\\ 
\vspace{0.2cm}
Since the Cylinder is inscribed in the sphere we get,
\begin{align}
h = 2\sqrt{R^2 - r^2}
\label{eq1}
\end{align}
We know that the Volume of the Cylinder is,
\begin{align}
V = \pi r^2 h
\label{eq2}
\end{align}
By substituting Eq. \eqref{eq1} in \eqref{eq2} we get,
\begin{align}
V = 2 \pi r^2 \sqrt{R^2 - r ^2}
\label{eq3}
\end{align}
Given to prove that $ h = \frac{2R}{\sqrt{3}} $ for $\vec{V_{max}}$. And to find the $\vec{V_{max}}$.\\
\vspace{0.25cm}
Hence, differentiating Eq. \eqref{eq3} w.r.t \textbf{r} we get,\\
\begin{align*}
\frac{dV}{dr} = \frac{d}{dr} [ 2 \pi r^2 \sqrt{R^2 - r ^2} ]
\end{align*}
By simplification we get,\\
\begin{align}
\frac{dV}{dr} = \frac{4 \pi r R^2 - 6 \pi r^3}{\sqrt{R^2 - r^2}}
\label{eq4} 
\end{align}
Now equating the Eq. \eqref{eq4} to zero and solving for \textbf{r} we get,\\
\begin{align}
r^2 = \frac{2R^2}{3}
\label{eq5}
\end{align}
Differentiating Eq. \eqref{eq4} once again w.r.t \textbf{r} and substituting \eqref{eq5} it is observed that,
\begin{align}
\frac{d^2V}{dr^2} \textless 0
\end{align}
Hence it is a point of Maxima.\\
\vspace{0.2cm}
Substituting Eq. \eqref{eq5} in Eq. \eqref{eq1},\\
\begin{align*}
h = 2 \sqrt{R^2 - \frac{2R^2}{3}}
\end{align*} 
\begin{align}
\therefore \vec{h} = \frac{2R}{\sqrt{3}}
\label{eq6}
\end{align}
Therefore, the Height of the cylinder of maximum volume that can be inscribed in a sphere of radius R is $\frac{2R}{\sqrt{3}} $.\\
\vspace{0.25cm}
\centering
Hence Proved.\\
\vspace{0.25cm}
\raggedright
Now, to find $V_{max}$, Substitute Eq. \eqref{eq6} in Eq. \eqref{eq2}. We get,\\
\begin{align*}
\vec{V_{max}} = \frac{4 \pi R^3}{3 \sqrt{3}}
\end{align*} 
\section{Verification Using Geomtric Programming}
\vspace{0.25cm}
\raggedright
Disciplined geometric programming, DGP is a subset of log-log-convex program (LLCP). An LLCP is defined as,\\
\vspace{2cm}
\hspace{2cm} minimize $f_0(x)$\\
\vspace{0.2cm}
\hspace{2cm} subject to $f_i(x) \leq \tilde{f}_i$, i = 1,2,...,m.\\
\vspace{0.2cm}
\hspace{3.8cm} $g_i(x) = \tilde{g}_i$, i = 1,2,...,p.
\begin{align}
\label{eq7}
\end{align}

where the functions $f_i$
				are log-log convex, $\tilde{f}_i$
 are log-log concave, and the functions $g_i$
				and $\tilde{g}_i$
 are log-log affine. An optimization problem with constraints of the above form in which the goal is to maximize or minimize a log-log concave function is also an LLCP. These LLCPs generalize geometric programming.\\

\vspace{0.25cm}

The given problem can be formulated as a DGP as,
\begin{align}
V = \max\limits_{r,h} \hspace{0.2cm} \pi r^2 h\\
s.t \hspace{0.5cm} h = 2\sqrt{R^2 - r^2}
\label{eq8}
\end{align}
By assuming R = 4 as input, and solving the above DGP Equations using Cvxpy we get,

\begin{align}
V_{max} = 309.55\\
h = 4.61\\
r = 3.26
\end{align} 

Hence it is proved that $ h = \frac{2R}{\sqrt{3}}$ for $V_{max}$.\\
\vspace{0.5cm}


\end{document}